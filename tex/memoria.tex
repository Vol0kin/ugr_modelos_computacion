\documentclass[11pt,a4paper]{article}
\usepackage[utf8]{inputenc}
\usepackage[english]{babel}
\usepackage{graphicx}
\usepackage{hyperref}
\usepackage{tikz}
\usepackage{makecell}
\usepackage{float}
\usepackage{enumitem}
\usepackage[textwidth=390pt]{geometry}

%Opciones de encabezado y pie de página:
\usepackage{fancyhdr}
\pagestyle{fancy}
\lhead{}
\rhead{}
\lfoot{Modelos de Computación}
\cfoot{}
\rfoot{\thepage}
\renewcommand{\headrulewidth}{0.4pt}
\renewcommand{\footrulewidth}{0.4pt}

\setlength{\parskip}{10pt}
\setlength{\parindent}{0pt}

\newcommand{\enu}{\textit{Enunciado}}
\newcommand{\sol}{\textbf{Solución}}
\newcommand{\noindentsect}{\setlength{\parindent}{0pt}}
\newcommand{\defaultindent}{\setlength{\parindent}{15pt}}

\begin{document}
	\pagenumbering{gobble}
	
	% Pagina de titulo
	\begin{titlepage}

		\begin{minipage}{\textwidth}

			\centering
			\textsc{\Large Modelos de Computación\\[0.2cm]}
			\textsc{GRADO EN INGENIERÍA INFORMÁTICA}\\[1cm]

			\noindent\rule[-1ex]{\textwidth}{1pt}\\[3.5ex]
			{\Huge Memoria de prácticas\\}
			\noindent\rule[-1ex]{\textwidth}{2pt}\\[3.5ex]
			%{\large\bfseries Ejercicio 5}
		\end{minipage}

		\vspace{1.5cm}
		
		\begin{minipage}{\textwidth}
			\centering

			\textbf{Autor}\\ {Vladislav Nikolov Vasilev}\\[2.5ex]

			\vspace{1cm}
			\textsc{Escuela Técnica Superior de Ingenierías Informática y de Telecomunicación}\\
			\vspace{1cm}
			\textsc{Curso 2018-2019}
		\end{minipage}
	\end{titlepage}
	
	% Indice
	\pagenumbering{arabic}
	\tableofcontents
	\newpage
	
	\section{Prácticas}
	
	\subsection{Práctica 1}
	
		\subsubsection{Ejercicio 1}
		\enu. Calcula una gramática libre de contexto que genere el lenguaje  
		$L = \lbrace a^n b^m c^m d^{2n}$ 
		tal que 
		$ n, m \geq 0 \rbrace$. \par
		
		\sol \par
		
		
		Se define la gramática como una cuádrupla con la forma $G = (V, T, P, S)$, siendo \textit{V} el conjunto
		de variables, \textit{T} el conjunto de elementos terminales, \textit{P} las reglas de producción y
		\textit{S} el símbolo inicial. Se puede definir cada uno de los conjuntos de la siguiente forma: \par
		\[V = \lbrace S, X, Y \rbrace\]
		\[T = \lbrace a, b, c, d \rbrace\]
		\[P = \lbrace S \rightarrow aXdd \; | \; bYc \; | \; \varepsilon, 
			   \; X \rightarrow aXdd \; | \; bYc \; | \; add \; | \; \varepsilon,
			   \; Y \rightarrow bYc \; | \; bc \; | \; \varepsilon \rbrace
		\]
		\[S = \lbrace S \rbrace \]
		\par
		
		Ésta es una gramática de \textbf{tipo 2}, ya que a la izquierda aparce una variable sola, sin símbolos
		terminales, y a la derecha aparece la variable con símbolos terminales tanto por la derecha como por la
		izquierda, impidiendo por tanto que sea regular por la izquierda o por la derecha. \par
		
		Con ésta gramática, primero se escoge si se van a empezar a generar una \textit{a} con la secuencia
		\textit{dd} al final, o si directamente se comenzará a generar la secuencia \textit{b} seguida de
		\textit{c}. Si se escoge la primera opción, se ponen tantas \textit{a} al principio y \textit{dd} al
		final como sea necesario, y después se puede escoger si se sigue con las \textit{b} y \textit{c}, o
		si se termina sin poner ninguno de los símbolos anteriores. Si se decide comenzar a poner \textit{b}
		y \textit{c} desde el principio o después de poner todas las \textit{a} y \textit{dd} que se quieran,
		se ponen todas las \textit{b} y \textit{c} que se quieran, hasta que se decida terminar la secuencia.
		\par
		
		Gracias a las reglas de producción se pueden satisfacer todas las restricciones del lenguaje, ya que
		por cada  \textit{a} se genera \textit{dd}, y por cada \textit{b} se genera \textit{c}. Además, se puede
		aceptar la cadena vacía o que alguna de las partes no esté.
		
		\subsubsection{Ejercicio 2}
		\enu. Describir una gramática que genere los números decimales escritos con el formato
		[signo][cifra][punto][cifra]. Por ejemplo, +3.45433, -453.23344, ...\par

		\sol \par
		La solución más sencilla que se puede ofrecer a este problema consiste en utilizar una gramática
		libre de contexto, como se mostrará a continuación. No obstante, el problema también es resoluble
		mediante una gramática regular, aumentanto sin embargo el número de producciones y de variables
		necesarias. \par
		
		Definimos la gramática como una cuádrupla con la forma $G = (V, T, P, S)$, siendo \textit{V} el conjunto
		de variables, \textit{T} el conjunto de elementos terminales, \textit{P} las reglas de producción y
		\textit{S} el símbolo inicial. Se puede definir cada uno de los conjuntos de la siguiente forma: \par
		
		\[V = \lbrace S, X \rbrace \]
		\[T = \lbrace 0, 1, 2, 3, 4, 5, 6, 7, 8, 9, ., +, -\rbrace \]
		\[P = \left\{\begin{array}{c}
    			S \rightarrow +X.X \; | \; -X.X \\
    			X \rightarrow 0X \; | \; 1X \; | \; 2X \; | \; 3X \; | \; 4X \; | \; 5X \; | \;
    			6X \; | \; 7X \; | \; 8X \; | \; 9X \; | \\ 
    			0 \; | \; 1 \; | \; 2 \; | \; 3 \; | \; 4 \; | \; 5 \; | \; 6 \; | \;
    			7 \; | \; 8 \; | \; 9
  			\end{array}\right\}
		\]
		\[S = \lbrace S \rbrace \]
		
		Primero se genera el signo y el punto, pudiendo escoger si el número es positivo o negativo. Después,
		en la parte entera y decimal se van generando números en el rango $[0, 9]$, pudiendo escoger cuál es
		el siguiente número o cuando terminar de insertar números.
				 
		
		\subsubsection{Ejercicio 3}
		\enu. Calcula una gramática libre de contexto que genere el lenguaje
		$L = \lbrace 0^i 1^j 2^k$
		tal que
		$i \neq j$ o $j \neq k \rbrace$. \par
		
		\sol \par
		
		Definimos la gramática como una cuádrupla con la forma $G = (V, T, P, S)$, siendo \textit{V} el conjunto
		de variables, \textit{T} el conjunto de elementos terminales, \textit{P} las reglas de producción y
		\textit{S} el símbolo inicial. Se puede definir cada uno de los conjuntos de la siguiente forma: \par
		
		\[V = \lbrace S, X, Y, Z, P, R, A, B, M, K, U, D, N, L, C, Q \rbrace \]
		\[T = \lbrace 0, 1, 2 \rbrace \]
		\[P = \left\{
		\begin{array}{c}
			S \rightarrow 0X1 \; | \; 0Y2 \; | \; 1Z2 \; | \; 0A1P2 \; | \; 0R1B2, \;
			X \rightarrow 0X \; | \; X1 \; | \; \varepsilon \\
			Y \rightarrow 0Y \; | \; Y2 \; | \; \varepsilon, \;
			Z \rightarrow 1Z \; | \; Z2 \; | \; \varepsilon, \;
			P \rightarrow P2 \; | \; \varepsilon, \;
			R \rightarrow 0R \; | \; \varepsilon, \\
			A \rightarrow 0M \; | \; N1, \;
			M \rightarrow OMU \; | \; \varepsilon, \;
			U \rightarrow 1 \; | \; \varepsilon, \;
			N \rightarrow CN1 \; | \; \varepsilon, \;
			C \rightarrow 0 \; | \; \varepsilon, \\
			B \rightarrow 1K \; | \; L2, \;
			K \rightarrow 1KD \; | \; \varepsilon, \;
			D \rightarrow 2 \; | \; \varepsilon, \;
			L \rightarrow QL2 \; | \; \varepsilon, \;
			Q \rightarrow 1 \; | \; \varepsilon
		\end{array}
		\right\}\]
		\[S = \lbrace S \rbrace \]
		
		Ya que hay muchas reglas y puede no llegar a quedar claro para qué es cada una, vamos a ir 
		comentándolas para que no queden dudas sobre el por qué de cada una de ellas. \par
		
		La primera de ellas, $S \rightarrow 0X1$, indica que solo se van a producir los símbolos
		0 y 1, dándose por tanto la condición $j \neq k$, ya que no hay ningún símbolo 2. $X$ puede ser
		sustituido por tantos 0 o 1 como se desee, lo cuál corresponde a la producción $X \rightarrow 0X \;
		| \; X1 \; | \; \varepsilon$. \par
		
		Después tenemos la producción $S \rightarrow 0Y2$, la cuál es parecida a la anterior, solo que
		esta vez se producen solo los símbolos 0 y 2, satisfaciendo por tanto las condiciones $i \neq j$
		y $j \neq k$ simultáneamente. La variable $Y$ puede ser sustituida por tantos 0 o 2 como se desee,
		lo cuál se corresponde con la producción $Y \rightarrow 0Y \; | \; Y2 \; | \; \varepsilon$. \par
		
		La regla $S \rightarrow 1Z2$ permite producir los símbolos 1 y 2. En este caso, esta regla permite
		satisfacer la restricción $i \neq j$, ya que no se produce ningún símbolo 0. La variable $Z$ puede
		ser sustituida por tantos 1 y 2 como se desee. Ésto se corresponde con la producción $Z \rightarrow 1Z
		\; | \; Z2 \; | \; \varepsilon$. \par
		
		
		\subsubsection{Ejercicio 4}
		\enu. Una empresa de videojuegos ``\textit{The fantastic platform}" están planteando diseñar
		una gramática capaz de generar niveles de un juego de plataformas, cada uno de los niveles
		siguiendo las siguientes restricciones:
		
		% Lista sin separación entre los elementos
		\begin{itemize}[noitemsep]
			\item Hay 2 grupos de enemigos: grupos grandes (\textit{g}) y grupos pequeños (\textit{p}).
			\item Hay 2 tipos de monstruos: fuertes (\textit{f}) y débiles (\textit{d}).
			\item Los grupos grandes de enemigos tienen, al menos, 1 monstruo fuerte y 1 débil.
				Y los 2 primeros monstruos pueden ir en cualquier orden. A partir del tercer
				monstruo, irán primero los débiles y después los fuertes.
			\item Los grupos pequeños tienen como mucho 1 monstruo fuerte.
			\item Al final de cada nivel habrá una sala de recompensas (\textit{x}).
		\end{itemize}
	
		% Se hace una seccion con un tamaño de letra pequeño
		\footnotesize
	
		Por ejemplo, la cadena terminal “\textit{gfddddfffpdddfx}” representa que el nivel tiene
		(\textit{gfddddfff}) un grupo grande con un monstruo fuerte, 4 débiles y otros 3 fuertes; 
		después tiene (\textit{pddddf}) un grupo pequeño con 3 débiles y uno fuerte. \par
	
		% Se restaura el tamaño de la letra al especificado en documentclass
		\normalsize	
		Elaborar una gramática que genere estos niveles con sus restricciones. Cada palabra del 
		lenguaje es un solo nivel. ¿A qué tipo de gramática dentro de la jerarquía de Chomsky 
		pertenece la gramática diseñada? \par
	
		¿Sería posible diseñar una gramática de tipo 3 para dicho problema?\par
	
		\sol \par
	
	\newpage
	\subsection{Práctica 2}
	
	\newpage
	\subsection{Práctica 3}
	
	\newpage
	\section{Ejercicios voluntarios}
\end{document}