\documentclass[11pt,a4paper]{article}
\usepackage[utf8]{inputenc}
\usepackage[english]{babel}
\usepackage{graphicx}
\usepackage{hyperref}
\usepackage{tikz}
\usepackage{makecell}
\usepackage{float}
\usepackage{enumitem}
\usepackage[textwidth=390pt]{geometry}

%Opciones de encabezado y pie de página:
\usepackage{fancyhdr}
\pagestyle{fancy}
\lhead{}
\rhead{}
\lfoot{Modelos de Computación}
\cfoot{}
\rfoot{\thepage}
\renewcommand{\headrulewidth}{0.4pt}
\renewcommand{\footrulewidth}{0.4pt}

\setlength{\parskip}{10pt}

\newcommand{\enu}{\textit{Enunciado}}
\newcommand{\sol}{\textbf{Solución}}
\newcommand{\noindentsect}{\setlength{\parindent}{0pt}}
\newcommand{\defaultindent}{\setlength{\parindent}{15pt}}

\begin{document}
	\pagenumbering{gobble}
	
	% Pagina de titulo
	\begin{titlepage}

		\begin{minipage}{\textwidth}

			\centering
			\textsc{\Large Modelos de Computación\\[0.2cm]}
			\textsc{GRADO EN INGENIERÍA INFORMÁTICA}\\[1cm]

			\noindent\rule[-1ex]{\textwidth}{1pt}\\[3.5ex]
			{\Huge Memoria de prácticas\\}
			\noindent\rule[-1ex]{\textwidth}{2pt}\\[3.5ex]
			%{\large\bfseries Ejercicio 5}
		\end{minipage}

		\vspace{1.5cm}
		
		\begin{minipage}{\textwidth}
			\centering

			\textbf{Autor}\\ {Vladislav Nikolov Vasilev}\\[2.5ex]

			\vspace{1cm}
			\textsc{Escuela Técnica Superior de Ingenierías Informática y de Telecomunicación}\\
			\vspace{1cm}
			\textsc{Curso 2018-2019}
		\end{minipage}
	\end{titlepage}
	
	% Indice
	\pagenumbering{arabic}
	\tableofcontents
	\newpage
	
	\section{Prácticas}
	
	\subsection{Práctica 1}
	
		\subsubsection{Ejercicio 1}
		\enu. Calcula una gramática libre de contexto que genere el lenguaje  
		$L = \lbrace a^n b^m c^m d^{2n}$ 
		tal que 
		$ n, m \geq 0 \rbrace$. \par
		
		\sol
		
		\subsubsection{Ejercicio 2}
		\enu. Describir una gramática que genere los números decimales escritos con el formato
		[signo][cifra][punto][cifra]. Por ejemplo, +3.45433, -453.23344, ...\par

		\sol
		
		 
		
		\subsubsection{Ejercicio 3}
		\enu. Calcula una gramática libre de contexto que genere el lenguaje
		$L = \lbrace 0^i 1^j 2^k$
		tal que
		$i \neq j$ o $j \neq k \rbrace$. \par
		
		\sol
		
		\subsubsection{Ejercicio 4}
		\enu. Una empresa de videojuegos ``\textit{The fantastic platform}" están planteando diseñar
		una gramática capaz de generar niveles de un juego de plataformas, cada uno de los niveles
		siguiendo las siguientes restricciones:
		
		% Lista sin separación entre los elementos
		\begin{itemize}[noitemsep]
			\item Hay 2 grupos de enemigos: grupos grandes (\textit{g}) y grupos pequeños (\textit{p}).
			\item Hay 2 tipos de monstruos: fuertes (\textit{f}) y débiles (\textit{d}).
			\item Los grupos grandes de enemigos tienen, al menos, 1 monstruo fuerte y 1 débil.
				Y los 2 primeros monstruos pueden ir en cualquier orden. A partir del tercer
				monstruo, irán primero los débiles y después los fuertes.
			\item Los grupos pequeños tienen como mucho 1 monstruo fuerte.
			\item Al final de cada nivel habrá una sala de recompensas (\textit{x}).
		\end{itemize}
	
	% Se hace una seccion sin indentar y con un tamaño de letra pequeño
	\footnotesize
	\noindentsect
	
	Por ejemplo, la cadena terminal “\textit{gfddddfffpdddfx}” representa que el nivel tiene
	(\textit{gfddddfff}) un grupo grande con un monstruo fuerte, 4 débiles y otros 3 fuertes; 
	después tiene (\textit{pddddf}) un grupo pequeño con 3 débiles y uno fuerte.\\
	
	% Se restaura el tamaño de la letra al especificado en documentclass
	\normalsize	
	Elaborar una gramática que genere estos niveles con sus restricciones. Cada palabra del 
	lenguaje es un solo nivel. ¿A qué tipo de gramática dentro de la jerarquía de Chomsky 
	pertenece la gramática diseñada?\\
	
	¿Sería posible diseñar una gramática de tipo 3 para dicho problema?\par
	
	\defaultindent
	\sol
	
	\newpage
	\subsection{Práctica 2}
	
	\newpage
	\subsection{Práctica 3}
	
	\newpage
	\section{Ejercicios voluntarios}
\end{document}